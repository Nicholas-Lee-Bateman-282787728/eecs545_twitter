% USAGE: copy contents of this file into abstractTitleIntroRelated.tex 

\begin{abstract}
We present two techniques to aid in the retrieval of information from Twitter. First we present a technique to cluster hashtags in meaningful topic groups using a combination of co-occurrence frequency, graph clustering and textual similarity. Second, we present a technique to classify a tweet in terms of these topic groups based on their word content using a combination of PCA dimensionality reduction and a variety of multi-class classification algorithms. We examine the relationship between the clustering step and the classification step to evaluate the performance of each.

\vspace{0.1in}
\begin{flushleft}
\textbf{Keywords:} Twitter, hashtag, machine learning, clustering, classification
\end{flushleft}
%
%\vspace{0.2in}
%\begin{center}
%\fontsize{13}{1}\textsc{EECS 545 Final Project, Fall, 2011}\selectfont
%\end{center}
\end{abstract}


\title{Classification of Tweets via Clustering of Hashtags}
%\author{Dolan Antenucci, Gregory Handy, Akshay Modi, Miller Tinkerhess} 
\author{\fontsize{12.5}{1}\textsc{EECS 545 Final Project, Fall, 2011}\selectfont
\vspace{0.2in} \\ \footnotesize{Dolan Antenucci, Gregory Handy, Akshay Modi,
Miller Tinkerhess } }


%\date{December 16, 2011 | EECS 545 Machine Learning Final Project}
\date{December 16, 2011}

\maketitle


\section{Introduction}
Twitter is the leading microblogging social network. It is the ninth most popular site on the Internet with over 200 million registered users producing over 200 million tweets every day \cite{Shiels2011,Alexa2011,Twitter.com2011}. Users post publicly viewable tweets of up to 140 characters in length, and follow other users whose tweets they are interested in receiving. The sheer volume of data produced by Twitter makes it an attractive area of study for machine learning. Unfortunately, many standard algorithms for extracting information from a body of text assume correct English. As a result, they are ineffective at analyzing tweets, which often contain slang, acronyms, or incorrect spelling or grammar.

Some words within a tweet are prefixed with punctuation symbols to indicate special meaning. For example, a word prefixed by ``$\#$'' is a {\it hashtag}. Hashtags are a way for a user to indicate the subject of a tweet in a way that is easy to search for; hashtags are deliberate metadata. We make the simplifying assumption that a tweet's hashtag content is a good approximation of its total content \cite{Rosa2011}. It follows that when multiple hashtags occur in the same tweet, they represent a similar approximation of content; the more often two hashtags co-occur, the more similar their meanings are.

In this paper we present an algorithm to learn the relationships between the literal content of a tweet and the types of hashtags that could accurately describe that content. As a classification problem, each tweet is represented as a frequency list of the non-stopword, non-hashtag words that appear in the tweet; its categories are the hashtags that are used in the tweet. In order to successfully classify tweets, we overcome problems of co-occurrence based graph clustering, dimensionality reduction, and multi-class categorization. We show that clustering algorithms and dimensionality reduction allow us to perform supervised classification on an otherwise intractable problem; that textual similarity measures may be used to expand graph-based clustering techniques without a resulting loss of precision; and that multi-class classification performs better than naive approaches on heavily compressed data.

Hashtag assignment could be used to suggest a tag to a user while they are composing a tweet, or to categorize untagged tweets as in semi-supervised learning. Hashtag clustering could also be used independently of classification, for example by inferring a user's topics of interest from the clusters to which their most frequently used hashtags belong; users could be shown other tweets, user profiles or advertisements that correspond to those interests.

\section{Related Work}
In his work on clustering hashtags, Poschko argues that two hashtags are similar if they co-occur in a tweet\cite{Poschko2011}. He creates a clustered graph with co-occurrence frequency as the distance measure. We expand upon his work by introducing a novel method for measuring the similarity between two hashtags. We use a larger set of hashtags and test several clustering methods instead of focusing on only one. 

Other work on clustering text-related entities typically focuses on a bag-of-words model that takes all the words of the entity followed by dimensionality reduction to make clustering computationally feasible \cite{Karandikar2010,Cheong2010}. We perform a similar dimensionality reduction to make classification feasible.

Davidov et al. attempt to classify the sentiment of a tweet in terms of its mood or opinion \cite{Davidov2010}. This requires a considerable amount of linguistic analysis. Our focus on classifying the topic of the tweet instead of the sentiment will not require this analysis, providing a simpler tool for classification.

Mazzia and Juett investigate the recommendation of hashtags based on a tweet's content using naive Bayes classification \cite{Mazzia2011}. We expand on this work by classifying a tweet into broader hashtag-cluster categories.